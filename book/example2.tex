\documentclass[12pt]{article}
\usepackage{amsmath}
\usepackage{amssymb}
\usepackage{CJKutf8}
\usepackage{ulem}
%\usepackage[normalem]{ulem}
\usepackage{textcomp}
\usepackage{siunitx}
\usepackage{lettrine}
\usepackage{indentfirst} %強制首段縮排
\usepackage{xcolor}
\usepackage{hyperref} %容易衝突,要放最後 p.37
\hypersetup{
	colorlinks=true,
	linkcolor=blue,
	filecolor=magenta,      
	urlcolor=cyan,
	pdftitle={Overleaf Example},
	pdfpagemode=FullScreen,
}
\setlength{\parindent}{1em} %水平長度(縮排)
\setlength{\parskip}{2pt plus 2pt minus 2pt} %垂直長度

\newenvironment{tightcenter}{%
	\setlength\topsep{0pt}
	\setlength\parskip{0pt}
	\begin{center}
	}{%
	\end{center}
}

\newenvironment{tightright}{%
	\setlength\topsep{0pt}
	\setlength\parskip{0pt}
	\begin{flushright}
	}{%
	\end{flushright}
}

\newenvironment{tightleft}{%
	\setlength\topsep{0pt}
	\setlength\parskip{0pt}
	\begin{flushleft}
	}{%
	\end{flushleft}
}

\title{The Title\thanks{ha ha ha}}
\date{October 10, 2021}
\author{OU, KAI-HAO\\CSIE\thanks{cmlab}, NTU}
\begin{document}
%\begin{CJK*}{UTF8}{bsmi} %細明體
\begin{CJK*}{UTF8}{bkai} %標楷體	
	\maketitle
	\begin{abstract}
	把摘要輕鬆帶過,顯然並不適合。摘要對我來說有著舉足輕重的地位,必須要嚴肅認真的看待。在人類的歷史中,我們總是盡了一切努力想搞懂摘要。
	\end{abstract}
	
	\section{大綱}
	
	這裡是大綱,也可輸入英文 \\
	You can also type english here.
	
	\LaTeX
	
	\LaTeXe
	
	\TeX
	
	測試! Hello world!
	
	\section{先備知識}
	
	查看 macro 的文檔:\\	
	[例如] texdoc ctex \\
	
	\noindent手寫符號辨識:Detexify \\ 
	https://detexify.kirelabs.org/classify.html \\
	
	\noindent公式截圖識別:Snip \\ 
	https://mathpix.com/ \\
	
	\section{CH3 latex基礎}
	
	$\backslash$  \\
	\textbackslash  \\
	\texttt{\char92}  \\
	\texttt{\char`\\}  \\
	\texttt{\char`~}  \\
	\texttt{\char`@}  \\
		
	\mbox{}\\
	a $\sim$ b  \\
	a\textasciitilde b  \\
	a\~{}b  \\
	a\~b  \\
	
	\mbox{}\\
	$>$ \\
	$<$ \\
	\textgreater \\
	\textless \\
	
	\mbox{}\\	
	`` `Max' is here.''。\\
	``\thinspace`Max' is here.''。\\
	
	\mbox{}\\	
	KAI-HAO\\
	page 1--2\\
	Listen---I'm serious.\\	
	中文省略號......\\
	英文省略號$\ldots$\\
	
	\noindent斜體表強調 \emph{gg gg gg} (斜體文本中,「正體」表強調) \\

	\mbox{}\\
	\uline{底線} \\
	\uuline{雙底線} \\
	\uwave{波浪線} \\
	\sout{刪除線} \\
	\xout{斜刪除線} \\
	\dashuline{虛 底線} \\
	\dotuline{點 底線} \\
	\uline{底\uuline{雙\xout{斜刪除線}底線}線} \\
	\emph{gg g\emph{gg gg gg}g gg} \\
	
	\mbox{}\\	
	\ang{30.33} \\
	$30^{\circ}\mathrm{C}C$ \\
	$30\,^{\circ}\mathrm{C}C$ \\
	$30\,$\textcelsius \\
	
	\newpage
	
	\mbox{}\\	
	歐元:\texteuro \\
	\mbox{1\,000\,000}\textdollar \\
	\mbox{1000\textperthousand} \\
	\mbox{\num[group-separator={,}]{1234567890}\textdollar}
	
	\begin{tightright}
	\=o
	\^o
	\d{o}
	$\tilde{o}$
	\o
	\aa
	\oe
	\'o
	\"o
	\u{o}
	\O
	\AA
	\OE
	\v o
	\.o
	\b{o}
	$\hat{o}$
	\i
	\ae
	!`
	\`o
	\H{o}
	\t{oo}
	\j
	\AE
	?`
	\end{tightright}
	
	\mbox{}		
	
	\begin{tightcenter}
	防止換行用\textasciitilde\\	
	防止換行用\textasciitilde
	\end{tightcenter}	
	
	Page 1000,  Page 1000,  Page 1000,  Page 1000,  Page 1000,  Page 1000,  Page 1000,  Page 1000,  Page 1000,  Page 1000,  Page 1000.
	
	Page~1000,  Page~1000,  Page~1000,  Page~1000,  Page~1000,  Page~1000,  Page~1000,  Page~1000,  Page~1000,  Page~1000,  Page~1000.
	
	\mbox{}
	
	\centerline{center text without adding space}
	
	\mbox{}
	
	帶有縮排的新段:段落文字文字,段落文字文字,段落文字文字,段落文字文字,段落文字文字。
	帶有縮排的新段:段落文字文字,段落文字文字,段落文字文字,段落文字文字,段落文字文字。
	\par帶有縮排的新段:段落文字文字,段落文字文字,段落文字文字,段落文字文字,段落文字文字。
	
	\mbox{}
	
	\lettrine{首字}{放大}的效果的效果的效果的效果的效果的效果的效果的效果的效果的效果的效果的效果的效果的效果的效果,的效果的效果的效果的效果的效果的效果的效果的效果的效果的效果的效果的效果。
	
	\mbox{}\\
	無襯線 \sffamily{sans-serif}\\
	等寬字 \ttfamily{Hello!}\\
	羅馬字 \rmfamily{Roman}\\	
	{\mdseries{中a8}粗Regular}\\
	{\bfseries{粗a8}體Bold}\\
	小號大寫體 \scshape{Caps and Small Caps}\\
	強調 \itshape{italic}\\
	斜 \slshape{slanted}\\
	直 \upshape{upright}\\
	
%	\tiny
%	\scriptsize
%	\footnotesize
%	\small
%	\normalsize (default)
%	\large
%	\Large
%	\LARGE
%	\huge
%	\Huge
	
	% inline
	\noindent\mbox{}\\
	{\fontsize{22}{\baselineskip}\selectfont Text Size Demo.\\}
	{\fontsize{23}{\baselineskip}\selectfont Text Size Demo.\\}
	{\fontsize{24}{\baselineskip}\selectfont Text Size Demo.\\}
	{\fontsize{50}{60}\selectfont Foo}
	{\fontsize{5}{6}\selectfont bar!}
	{\Huge Foo}
	{\tiny bar!}
	{\Large This is some large text\par}
	% environment
	\begin{footnotesize}
		text size environment...
	\end{footnotesize}

	Let's change font to {\fontfamily{lmss}\selectfont Palatino} % p.32 表3.5
	
	\noindent\mbox{}\\
	文字顏色:\\
	{\textcolor{pink}{TXT:pink.}} \\
	{\textcolor{red}{TXT:red.}} \\
	{\textcolor{orange}{TXT:orange.}} \\
	{\textcolor{yellow}{TXT:yellow.}} \\
	{\textcolor{lime}{TXT:lime.}} \\
	{\textcolor{green}{TXT:green.}} \\
	{\textcolor{olive}{TXT:olive.}} \\
	{\textcolor{teal}{TXT:teal.}} \\
	{\textcolor{cyan}{TXT:cyan.}} \\
	{\textcolor{blue}{TXT:blue.}} \\
	{\textcolor{violet}{TXT:violet.}} \\
	{\textcolor{purple}{TXT:purple.}} \\
	{\textcolor{magenta}{TXT:magenta.}} \\
	{\textcolor{brown}{TXT:brown.}} \\
	{\textcolor{white}{TXT:white.}} \textless- white\\
	{\textcolor{lightgray}{TXT:lightgray.}} \\
	{\textcolor{gray}{TXT:gray.}} \\
	{\textcolor{darkgray}{TXT:darkgray.}} \\
	{\textcolor{black}{TXT:black.}} \\
	\mbox{}\\
	{\textcolor{red!70}{70\%紅色}} \\
	{\textcolor{-yellow}{黃色的互補色}} \\
	
	\newpage

	\subsection{CH3-5 引用與註釋 (p.37)}



	% chapter section figure table equation code listing

	\label{figure:test}	
	See figure~\ref{figure:test} on page~\pageref{figure:test}.
	
	\label{section:XXX}
	See section~\ref{section:XXX} on page~\pageref{section:XXX}.

	we have solved equation \eqref{eq:QQQ} %\ref{eq:QQQ} 
	
	
	\mbox{}\\
	{\Large Hyperref} \\	
	\hyperref[section:XXX]{GO TO section:XXX}\\
	The equation \ref{eq:QQQ} shows in the page \pageref{eq:QQQ}.\\	
	HyperLink:  \href{http://www.overleaf.com}{Something Linky}\\
	 or go to the next url: \url{http://www.overleaf.com}\\
	 or open the next file \href{run:./file.txt}{File.txt}\\
	 
	幫我來個註腳\footnote{這是個註腳。}\\
	
	來個邊註\marginpar[左左]{右右}\\

	\begin{quote}
		引用文字,引用文字;\\
		引用文字,引用文字。
		
		引用文字,引用文字;\\
		引用文字,引用文字。
	\end{quote}
	
	\begin{quotation}
		引用文字,引用文字;\\
		引用文字,引用文字。
		
		引用文字,引用文字;\\
		引用文字,引用文字。
	\end{quotation}


	
	
	測試! Hello world! 氣溫
	\newpage
	\section{CH4 latex數學排版}
	
	\begin{equation} \label{eq:QQQ}
		x^2 - 5 x + 6 = 0
	\end{equation}

	${A^{\dagger}}$
		
	$\begin{pmatrix} 
		\text{氣溫}_1 & \text{氣溫}_n \\
		\text{濕度}_1 & \text{濕度}_n 
	\end{pmatrix}$\par

	
	\newpage\mbox{}\newpage\mbox{}\newpage
	
	an article~\cite{RN42}
	
	\bibliographystyle{plain} % Choose style for bibliography (選擇引文的格式)	{可選:prsty / abbrv / alpha / plain / unsrt}
	\bibliography{refxxx}        % ref.bib is the name of our database
	
\end{CJK*}
\end{document}